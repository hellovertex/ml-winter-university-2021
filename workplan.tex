\documentclass[a4paper, 11pt]{article}

\usepackage{geometry}
\geometry{a4paper,left=30mm,right=30mm, top=35mm, bottom=30mm}

\usepackage[ngerman]{babel}
\usepackage[utf8]{inputenc} 
\usepackage[T1]{fontenc}
\usepackage{algorithm}
\usepackage{amsmath}
\usepackage{amssymb}
\usepackage{amsthm}
\usepackage{enumerate}
\usepackage{array}
\usepackage{wasysym}
\usepackage{fancyhdr}
\usepackage{graphicx}
\usepackage{hyperref}
\usepackage{tabularx}
\usepackage {tikz}
\usetikzlibrary {positioning}
%\usepackage {xcolor}
\definecolor {processblue}{cmyk}{0.96,0,0,0}
\usepackage{latexsym}
\usepackage{lastpage} % Seitenzahlen
%\usepackage{MNsymbol}
\pagestyle{fancy}
\usepackage[noend]{algpseudocode}
\usepackage{caption}
\usepackage{amsmath}
\usepackage{tikz}
\usetikzlibrary{arrows}
\usepackage{tabularx} %schöne tabellen
\parindent0pt %einrücken verhindern
\bibliographystyle{unsrt}

\usepackage{polynom}
\cfoot{\thepage  \ / \pageref{LastPage}}

% % % % % % % % % % % % % % % % % % % % % % % %
% % % % % % % % % % % % % % % % % % % % % % % %
\newcommand{\modullang}{BAs}
\newcommand{\semester}{SoSe 2018}
\newcommand{\modul}{BA}
\newcommand{\blatt}{}
\newcommand{\tutorium}{Mi 16-18 Uhr}
\newcommand{\tutor}{}
\newcommand{\datum}{\today}
\renewcommand{\proofname}{Proof}
\newcolumntype{C}{>{\Centering\arraybackslash}X}
% % % % % % % % % % % % % % % % % % % % % % % %
% % % % % % % % % % % % % % % % % % % % % % % %

\begin{document} 
	
	%%% Kopfzeile linker Bereich
	%      gerade Seite   ungerade Seite
	\rhead[ \leftmark   ]{\textbf{}}
	%%% Kopfzeile mittlerer Bereich
	%      gerade Seite   ungerade Seite
	\chead[\leftmark   ]{\leftmark{}}
	%%% Kopfzeile linker Bereich
	%      gerade Seite             ungerade Seite
	\lhead[\textbf{}]{\blatt}
	
	
	%-- Deckblatt --						      
	\thispagestyle{empty}
	\begin{center}
		
		\vspace*{1.4cm}
		{\LARGE \textbf{Technische Universität Berlin}}
		
		\vspace{0.5cm}
		
		{\large Master Thesis outline: Informal draft\\[1mm]}

		
		
		\vspace{1.0cm}
		{\LARGE \textbf{Title: tba}}\\
		\vspace*{1.0cm}
		
	
		Sascha Lange%, 349960
		
		
		
	\end{center}
	
	\renewcommand{\labelenumi}{\alph{enumi})}
	\renewcommand{\labelenumii}{(\roman{enumii})}
	\renewcommand{\labelenumiii}{\arabic{enumiii}.}
	\renewcommand{\contentsname}{Table of Contents}
	%\renewcommand{\labelenumii}{\textbf{-}}	
	



%-- Eigentlicher Text --
	\newpage
	\newtheorem{Cor}{Corollary}
	\newtheorem{Theorem}{Theorem}
	\newtheorem{Def}{Definition}
	\newtheorem{Prop}{Proposition}
	\newtheorem{Lemma}{Lemma}
	\section*{Outline of steps}
% IDEA:
% 1. Train pool of mini models
% [optional] build similarity matrices that form the agreement tensor	
% 2. meta learn a network that can generate mini models using a pool of states
% I am creating a draft for the working plan until next week.

\begin{tabularx}{\textwidth}{|l|X|X|}
    step & informal & formal \\
\hline
    1 & use env. to generate states 
		(all expert levels, requires $\geq 1$  good agents)  
	& sample from DEC-POMDP \\
\hline
    2 & Obtain pool of agents 
	& Q-function estimation, policy gradient, ruleset\\
\hline
    3 & Train minimodel for each agent (predicts agents action given state) & neural-net function approximator $N_i: \mathcal{S} \rightarrow \mathcal{A}$ for each agent $i \in POOL$\\
\hline
   4 & Compute Meta-mini-model by considering tasks as subset of agents $\subset POOL$ (Optional: hypernetwork approach, evolutionary approach) & using \href{https://proceedings.mlr.press/v70/finn17a/finn17a.pdf}{Alg.1 from MAML} where $\mathcal{T}_i \subset{POOL}$ \\
\hline
  5 & Proof of concepts for i) simple environemnts (grid, graph)  ii) chess? iii) hanabi small, mid, normal & - \\
\hline
  6 & Test adhoc performance of meta mini model for unseen player & -\\
\hline 
\end{tabularx}

Basically, step 2 requires Reinforcement Learning setup, step 3 requires deep supervised learning, and step 4 requires meta learning. \\

Once steps 1 to 5 are established, we can aid the meta learning process by introducing additional metrics/architectures/paradigms to step 4. 

I have handbuild a baseline to test the performance of step 4 against that does not require deep learning but its still under development.
\end{document}